\documentclass[a4paper, 12pt]{article}
\usepackage[T1]{fontenc}
\usepackage{lmodern}
\usepackage{libertine}
\usepackage{color}
\definecolor{darkblue}{rgb}{0.0, 0.0, 0.55}
\usepackage{soul}
\usepackage{setspace}
\usepackage[left=2.25cm, top=2.5cm, right=2cm, bottom=2cm]{geometry}
\usepackage{hyperref}
\usepackage{graphicx}
\usepackage[english]{babel}
\usepackage[authoryear]{natbib}
\usepackage[german=guillemets]{csquotes}
\usepackage{bookmark}
\usepackage{url}
\usepackage{etoolbox}

\makeatletter

% make numeric styles use name format
\patchcmd{\NAT@test}{\else \NAT@nm}{\else \NAT@nmfmt{\NAT@nm}}{}{}

% define \citepos just like \citet
\DeclareRobustCommand\citepos
  {\begingroup
   \let\NAT@nmfmt\NAT@posfmt% ...except with a different name format
   \NAT@swafalse\let\NAT@ctype\z@\NAT@partrue
   \@ifstar{\NAT@fulltrue\NAT@citetp}{\NAT@fullfalse\NAT@citetp}}

\let\NAT@orig@nmfmt\NAT@nmfmt
\def\NAT@posfmt#1{\NAT@orig@nmfmt{#1's}}

\makeatother

\exhyphenpenalty=1000
\hyphenpenalty=1000
\widowpenalty=1000
\clubpenalty=1000

\hypersetup{breaklinks=true,
            pdfauthor={David Skarbek and Danilo Freire},
            pdftitle={Prison Gangs},
            colorlinks=true,
            citecolor=darkblue,
            urlcolor=darkblue,
            linkcolor=darkblue,
            pdfborder={0 0 0}}

\title{Prison Gangs}

\author{
David Skarbek\thanks{David Skarbek is Senior Lecturer in Political Economy at King's College London. He received a PhD in Economics from George Mason University, has written extensively on organized crime groups, and his most recent book is \textit{The Social Order of the Underworld: How Prison Gangs Govern the American Penal System}. Email address: \href{mailto:david.skarbek@kcl.ac.uk}{david.skarbek@kcl.ac.uk}. Website: \href{http://www.davidskarbek.com/}{http://www.davidskarbek.com}. } 
\and 
Danilo Freire\thanks{Danilo Freire is a PhD Candidate in the Department of Political Economy at King's College London. His main research interests are prison gangs, urban crime in Latin America, comparative political violence, and causal inference designs. He gratefully acknowledges the support of the Brazilian National Council for Scientific and Technological Development and the Faculty of Social Science and Public Policy at King's College London. Email address: \href{mailto:danilofreire@gmail.com}{danilofreire@gmail.com}. Website: \href{http://danilofreire.com}{http://danilofreire.com}. } 
}

\date{June 25, 2016}

\begin{document}

\maketitle

\doublespacing

\begin{abstract}

\noindent Although widely seen as unruly and predatory, prison gangs operate as quasi-governments in American correctional facilities. Inmate groups enforce property rights, regulate illicit markets, and promote cooperation when the state is unable or unwilling to act. Prison gangs are relatively new to the United States, and are best understood as unintended consequences of recent shifts in inmate demographics and the gradual erosion of the convict code; similar factors have caused the proliferation of prison gangs worldwide. Inmate institutions have emerged in countries as diverse as Sweden and Bolivia, and they exhibit considerable variation in size and scope. The impact of prison gangs on street-level criminal activities and directions for further research are also discussed.

\vspace{.5cm}

\noindent \textsc{Keywords}: criminal organizations; extralegal institutions; prison gangs; private governance; rational choice \\

\vspace{.25cm} 

\begin{center}
\noindent Forthcoming in Hayden Griffin and Vanessa Woodward (eds). 2016.\\ \textbf{Handbook of Corrections in the United States}. London: Routledge. 
\end{center}


\end{abstract}

\newpage

\section{Introduction}
\label{sec:intro}

\noindent Prison gangs have radically changed the dynamics of the United States penal system. These are  inmate organizations that exist into perpetuity, and whose membership is restrictive, mutually exclusive, and often requires a lifetime commitment.\footnote{This is related to \citepos[p. 48]{lyman1989gangland} definition of a prison gang as an ``organization which operates within the prison system as a self-perpetuating criminally oriented entity, consisting of a select group of inmates who have established an organized chain of command and are governed by an established code of conduct.'' While often descriptively accurate, we believe this definition to be too narrow; some prison gangs do not have a strict hierarchy, such as the Mexican Mafia, and others are not actively engaged in criminal activity, only in protection functions.} Prior to the 1950s, prison gangs did not exist in the country, but by the late 1970s, inmate organizations were already a dominant force in American correctional facilities \citep[e.g.][]{fleisher2001overview,howell2015history,skarbek2014social,wells2002study}. The strength of these groups can be inferred from their membership numbers. In 1985, there were about 113 gangs with 13,000 active members in American prisons \citep{camp1985prison}. By 2002, in contrast, about 308,000 prisoners were affiliated with inmate groups \citep{winterdyk2010managing}. The corrections director of California has attested that in 2006 there were up to 60,000 gang members in that state alone \citep{petersilia2006understanding}. As these numbers do not include people who are indirectly involved with prison activities -- such as visitors who smuggle narcotics into prisons for felons to trade \citep{crewe2006prison,lessing2014build} -- the real influence of inmate gangs is probably more extensive than official figures suggest.

The emergence of prison gangs was far from peaceful. Gangs have been responsible for most cases of serious misconduct in jails, such as inmate assault \citep{cunningham2007predictive, ralph1991gang,reisig2002administrative}, staff intimidation \citep{gaes2002influence}, sexual misbehavior \citep{ralph1991gang,wyatt2005male}, and drug trafficking \citep{shelden1991comparison}. Moreover, in recent years, prison groups have expanded their reach and made inroads into street-level drug markets,  repeatedly resorting to force \citep[e.g.][]{bony2015prison,lessing2014build,skarbek2011governance,valdez2005mexican}. For all these reasons, it is unsurprising that inmate organizations are now regarded as the most serious threat to the American prison administration by staff and academics alike \citep{carlson2001prison,delisi2004gang,fleisher2001overview}.

But despite their history of violence, prison gangs are not disorganized collectives. In fact, many gangs are highly structured institutions, often with strict hierarchies, elaborate internal rules, and comprehensive sets of norms \citep{leeson2010criminal,skarbek2012prison}. Most importantly, gangs provide what the prison setting sometimes lacks: \textit{social order}. The literature tells us that the private supply of public goods is not only theoretically possible \citep{bergstrom1986private,olson1965logic,ostrom1992covenants}, but in fact is a frequent occurrence, with such public goods provided by a myriad of social groups. From commodity producers \citep{schepel2005constitution} to Chinese warlords \citep{jackson2003warlords}, a variety of institutions have developed their own enforcement mechanisms to punish noncompliant behavior and promote cooperation. Prison gangs are no exception. 

Self-governing groups are commonly expected to define and secure property rights \citep{gambetta1996sicilian,skaperdas2001political,varese2011mafias}. Since inmates are constantly subject to extortion and violence, there is high demand for security from the incarcerated population. Convicts cannot always rely on prison staff for protection -- correctional officers may have limited resources, limited information, or both -- therefore they often turn to extralegal institutions for help. Evidence shows that gangs have been successful at protecting property and, perhaps surprisingly, their rise to power has coincided with a dramatic fall in victimization in prisons; the number of inmate riots, assaults, homicides, and suicides have all decreased over recent decades \citep{useem2006prison}. Paradoxically, violent gangs are making prisons safer.

Prison gangs also help inmates to make gains from trade. Although the state actively discourages illegal commerce among criminals, trade is widespread in the penal system \citep{davidson1974chicano,kalinich1986power,lankenau2001smoke,williams1974convicts}. The contraband marketplace is so important to inmate social life that some authors call it ``the basis of legitimate power'' within prisons \citep{kalinich1985contraband}. Nevertheless, market transactions are costly in jails. By the nature of their own business, criminals generally distrust one another \citep{gambetta2009codes}. Prison groups solve this social dilemma by enforcing contracts (through violence if required), monitoring transactions, providing general understanding of trade rules, and contacting potential suppliers of goods from street gangs \citep{blatchford2008black,skarbek2012prison}. Gangs have created the prisoners' equivalent of the \textit{lex mercatoria}, the famed commercial law used by merchants across Europe during the medieval period \citep{coquillette1998lex,lando1985lex}.

Although strongly associated with American prisons, inmate groups are not exclusive to the United States. Gangs govern the penal system in large parts of the world. In Latin America, for instance, prison gangs have informally controlled correction facilities for decades. Bolivia's San Pedro Prison is an example of a self-governing community, wherein prisoners have maintained orderly coexistence even though state authorities have only a minimal presence in the prison \citep{cerbini2012sanpedro,skarbek2010self}. The \textit{maras} in El Salvador also have considerable power over the country's prisons. As many important Salvadoran gang leaders are currently behind bars, some street gangs are now managing their operations from jails \citep{cruz2006maras,wolf2012mara}. The \textit{Primeiro Comando da Capital} (PCC), a Brazilian inmate organization, is another illustrative case. The PCC is reportedly present in 22 of the 27 Brazilian states, makes profits of roughly 50 million dollars per year, and has allegedly managed to have their own representatives elected in the country's most recent elections \citep{biondi2008etica,veja2013}. 

Western European countries have experienced a similar rise in prison gang activities. One of these new groups is the \textit{Brödraskapet}, ``The Brotherhood''. The Brödraskapet is a Swedish inmate gang with several chapters across the country.\footnote{The gang even has its own website: \href{http://www.brodraskapet.se}{http://www.brodraskapet.se}. Access: March 27, 2016.} Albeit small, the gang allegedly counts Sweden's most dangerous criminals among its ranks and has been involved in drug trafficking, extortion and murder \citep{larsson2011svensk}. In the United Kingdom, Muslim prison gangs are increasingly influential. Muslims, who make up about 3\% of the total British population, account for 15\% of convicts in the UK \citep{telegraph2014muslimgangs} and for a third of prisoners in the Category A (high-security) Whitemoor Prison \citep{BBC2010muslimgangs,guardian2008muslim}. A 2010 report by the HM Chief Inspector of Prisons for England and Wales indicates that inmates are converting to Islam not only because of the comfort of the religion, but also for ``[an] opportunity to obtain support and protection in a group with a powerful identity; and [\dots] by perceptions of the material advantages of identifying as Muslim'' \citep[p. 7]{hm2010muslim}. As stated by an inmate from Whitemoor Prison, ``[\dots] it's basically like a protection racket, that's how it runs, `we can offer you security, if ever anybody threatens you, we'll sort it', but you've got to become a Muslim'' \citep[p. 69]{liebling2011exploration}. 

In spite of their growing importance, scholars have paid little attention to prison organizations. Academic works on street gangs greatly outnumber books and articles on prison gangs,\footnote{A quick search on Google Scholar shows that there are about 22,700 results for ``street gangs'' (\href{https://goo.gl/QMdBTU}{https://goo.gl/QMdBTU}) but only 3,130 for ``prison gangs'' (\href{https://goo.gl/N4LuXR}{https://goo.gl/N4LuXR}). Access: March 27, 2016.} and with few exceptions \citep[e.g.][]{freire2014entering,lessing2014build,skarbek2010self,skarbek2011governance,skarbek2012prison,skarbek2014social}, inmate institutions have been virtually ignored by political scientists and economists. Yet these two disciplines can offer valuable insights into the inner workings of criminal groups. Rational choice theory, widely employed by economists, appears particularly suited to this task. Rational choice is a variant of methodological individualism \citep{arrow1994methodological,blaug1992methodology}, and one of its basic premises is that macrobehavior can be explained by the purposive actions of self-interested individuals \citep{friedman1988contribution,udehn2002methodological}. The theory does not require agents to have complete information or perform perfect calculations of their pay-offs \citep{simon1955behavioral}, but it assumes that an individual will try to maximize their utility whenever possible \citep{neumann1947theory,friedman1948utility}.

Criminals are particularly inclined to behave rationally, as their environment forces them to do so. Mistakes are severely punished in  jails. Errors in judgment may lead to death. Hence we see rational choice as a useful framework to analyze social preferences and collective outcomes in prisons. Furthermore, rational choice does not disregard the role social norms play in the formation of individual preferences \citep{crawford1995grammar,elster1989social,ostrom2014collective}. As we argue hereafter, shared perceptions have framed prison gangs since the earliest days of the phenomenon. However, if the inmate community grows larger and more diverse, norms have to be supplemented by other arrangements such as an organization \citep{skarbek2012prison}. The rational choice framework can integrate these various mechanisms into a single, cohesive theory of gang behavior. 

The text proceeds as follows. In Section Two, we discuss how California prison gangs use the community responsibility system to enforce rules. In Section Three, we explain the decline of the convict code and show that changes in prison demographics generated the conditions for a different type of inmate groups. Section Four focuses on the ways by which prison gangs influence street-level criminal markets. Section Five details prison gang governance in another countries, specially in Northern Europe and Latin America. Section Six concludes and offers possible topics for further research. 

\section{How Gangs Operate}
\label{sec:gangsoperate}

\noindent
What drives gang formation? The media generally portray prison gangs as racist organizations whose main goal is protection against rival ethnic groups. The idea does not seem far-fetched: many American gangs, such as the Aryan Brotherhood, the Black Family, or the Mexican Mafia, are indeed organized along racial lines \citep{fong1990organizational,hunt1993change,pelz1991right}. But while race does play a role in gang recruitment, ethnic competition is not the key factor behind gang growth. Rather, we argue that prison gangs are created essentially to foster contraband markets through the promotion of cooperation and trust between inmates \citep{fleisher2001overview,roth2014prison}. 

This view is consistent with a vast literature on self-enforcing exchange. Several authors claim that it is possible for decentralized communities to engage in trade even without the presence of strong government institutions \citep[e.g.][]{dixit2007lawlessness,friedman1989machinery,leeson2007efficient,ostrom1992covenants,powell2009public,stringham2005anarchy,tullock1972explorations,tullock1974further}. Most people in the world still live under governments that are ineffective, weak or corrupt, and many firms run their businesses in areas where the state has only imperfect, if any, control \citep[p. 504]{powell2009public}. But how can we have governance without governments?

The academic criticism of anarchy is based upon the assumption that if state regulations were absent, long-term exchange could not persist because every interaction would be characterized as a prisoner's dilemma. That is, even if both parties could gain from cooperation, they would still have an incentive to cheat due to the lack of external enforcement of property rights \citep{bush1972individual,buchanan1975limits,mueller1988anarchy,tullock1972edge}. However, in reality, many self-organizing groups devise private mechanisms to prevent predatory behavior. Historical examples abound. \citet{leeson2007arrgh,leeson2007efficient,leeson2009rationality,leeson2009calculus,leeson2010pirational} describes how late seventeenth- and early eighteenth-century pirates used reputation strategies to maximize profits. \citet{stringham2015private} argues that stock exchange traders employed club membership as a signal of trustworthiness. \citet{de1990other}, in turn, analyzes the informal system of property rights in modern Peru and shows how a thriving illegal economy can subsist without, and sometimes in confrontation with, state institutions. 

These cases demonstrate that seemingly erratic behavior may just be rational responses to unusual economic incentives \citep[p. 6]{leeson2009invisible}. Prison gangs can also be understood through such a lens. Social coordination in gangs is often achieved with a \textit{community responsibility system} (CRS). This institutional device was first employed by merchants in the late medieval period in Europe, and it comprises of a system where the whole community is responsible for the actions and debts of their individual members \citep{greif2006institutions}. A simple example may clarify how CRS induces trustworthiness: 

\begin{quote}
Consider a situation where a member of Group A borrows money from a member of Group B. If Member A defaults on the debt, then all members of Group A are responsible for repaying it. If Group A does not suitably compensate Member B, then Group B boycotts Group A. If there are substantial benefits available from future interactions with Group B, then the threat of boycott induces payment by Group A. [\dots] Moreover, the corporate nature of the group creates a repeated play scenario among groups even though particular members may never trade again. When groups have reputations for taking responsibility for its members' actions, then two members of different groups who do not know each other can still benefits from trade. \citep[p. 226]{roth2014prison}
\end{quote}

Qualitative evidence indicates that gangs indeed operate within this type of system. There are two conditions for CRS to work in prisons (and elsewhere). First, individuals should be able to signal their group affiliation, so that the trading partner knows that CRS will be enforced on that specific person. This type of screening is not difficult in jail, as inmates routinely use costly signals to convey information \citep{gambetta2009codes}. On the one hand, race usually denotes gang membership in prisons, and this is a signal that is impossible to fake. On the other hand, inmates are eager to display voluntary signals of gang affiliation: language use, gestures, tattoos, just to name a few \citep{kaminski2010games,valentine2000gangs}. Salvadoran gangs, for instance, are famous for their widespread use of large facial tattoos, a very credible signal of group commitment  \citep{cruz2010central,wolf2010maras}.

Second, the community must be able and willing to punish misbehaving members. This condition is also met in the penal system. Gangs routinely use force (or the threat thereof) to maintain social order and punish defectors \citep{skarbek2011governance,skarbek2012prison,skarbek2014social}. A Californian inmate interviewed by \citet[p. 763]{trammell2009values} illustrates this point: ``if one of my guys is messing up then we either offer him up to the other guys or we take him down ourselves.'' The same happens in other parts of the world. Some members of the \textit{Primeiro Comando da Capital}, Brazil's largest prison gang, are explicitly designated to take care of group affiliates who do not obey their rules. Predictably, they are called \textit{disciplinas} (`punishers', literally `disciplines') \citep{biondi2010junto,dias2009guerra}. Although in many cases the PCC member has the right to defend himself in a sort of ``criminal court,'' the \textit{disciplina} is authorized by the gang to inflict any required punishment on the offending member \citep{dias2011pulverizaccao,marques2010liderancca}. This signals to the other parties that the group is reliable. At the same time, the punishment reinforces cohesion and adherence to the gang's internal rules. 

What is remarkable about CRS is that it facilitates trade even if individuals are not of a cooperative type. This is due to low monitoring costs \citep{biais1997trade,carletti2004structure,jain2001monitoring}. Members of the same group tend to have better information about each other, so it is easy to find and punish an individual who breaks the rules \citep{greif2006institutions}. This creates incentives for communities -- even communities that may be rivals in other contexts -- to engage in between-group trade. Prison gangs may be divided by race, but inmates apparently do not let this factor interrupt exchange flows. As noted by a convict in California, ``the races don't officially mix. That's true but you can buy drugs from whoever and the leaders control that stuff. [\dots] It's not as cut and dry as you think'' \citep[p. 756]{trammell2009values}. Racial tensions could easily escalate in prisons, but because gang wars are costly, groups have an incentive to be peaceful. Order is good for business.\footnote{This theory is also akin to the ``capitalist peace'' framework of international relations, although the specific causal mechanisms are clearly different. The theory suggests that trade reduces the propensity of war because states which have mutual commercial interests will both lose if a war erupts \citep[e.g.][]{gartzke2007capitalist, kant1795ewigen, levy2011causes}.}

This system becomes established because trade structures inmate relations. Goods that are easily accessible to the general population are notably scarce in prisons. Access to them therefore lends status and prestige to prisoners. Paul, a black British inmate in his early 30s, describes the role illegal trade plays in prisoner hierarchy: ``When I was [dealing] I could say: `I'm a top dog. I've got drugs, I've got this, I've got that, yeah, no-one can't fuck with me'. [\dots] Drugs is power in here, yeah, so is tobacco, and without drugs, tobacco and phonecards [prisons] don't really work.'' \citep[p. 360--361]{crewe2006prison}. 

Prison gangs are key players in the contraband markets. As a prison official notes, ``almost without exception [\dots] the gangs are responsible for the majority of drug trafficking in their institutions'' \citep[p. 52]{camp1985prison}. \citet[p. 361--362]{crewe2006prison} quotes a dialogue with one interviewee where the inmate reflects on the link between money from illegal trading and group protection:

\begin{quote}
``\textit{If you have the drugs but you have no violence, does that mean the drugs just get taken off you?}''

``[\dots] You need backing. You yourself don't need violence. You've got bounty hunters in prison. [\dots] People who, for a price, will protect you. [\dots] Any smart person would get linked up with the right group. [\dots]''

``\textit{So you're saying that people then gang together because it's a form of protection?}''

``Yes, it's a form of protection and it's power. If I've got half an ounce of heroin I can turn that into probably three or four grams, that's a lot of money in prison, and if you're keeping two or three guys sweet with you, they don't want that breaking up. They're thinking, `fucking hell, we're living alright, we've got it easy in here, nobody is fucking up our little crew, we're sticking together'.''
\end{quote}

This can be generalized to larger groups. \citet[p. 755]{trammell2009values} writes how Jack, an inmate in a California jail, explains the role of prison groups:

\begin{quote}
``The boys inside, they follow the rules and that means you work with your own boys and do what they say. Look, there is a lot of problems caused by the gangs, no doubt. The thing is, they solve problems too. You want a structure and you want someone to organize the businesses so the gangs have their rules. You don't run up a drug debt, you don't start a fight in the yard and stuff. Gangs are a problem but we took care of business.''
\end{quote}

In summary, the fundamental role of prison gangs is to promote cooperation between inmates who have strong reasons to distrust each other. This is done to achieve an important goal: trade. As prisoners live in a resource-scarce world, trading acquires a significant importance, not only in terms of the gains it may bring to dealers, but also through the social relationship it forges. The community responsibility system ensures that commercial exchanges will not be interrupted by predatory individuals. 

\section{The Decline of the Convict Code}
\label{sec:convictcode} 

\noindent
Before prison gangs and the community responsibility system, the main source of inmate governance in California was a set of informal norms known as ``the convict code'' \citep{irwin1962thieves,irwin1970felon,jacobs1977stateville,sykes1960inmate}. The code relies on strong images of masculinity \citep{freeman1999correctional,hua2005patterns} and emphasizes the importance of being tough, and sometimes hostile, toward fellow prisoners and staff \citep[pp. 369]{cole2013criminal}. Although the code does not include a fixed list of rules -- its application varies significantly from case to case  \citep[e.g.][]{akers1977prisonization,copes2013accounting,trammell2012enforcing} -- \citet[p. 5--9]{sykes1960inmate} affirm that its chief tenets may be classified into five groups. First, there are norms that suggest caution to felons, and are usually condensed in the maxims, ``Do not interfere with inmate interests'' or ``Do rat on an inmate.'' These suggest that inmates should serve their time as freely as possible, with the minimum amount of interference from other prisoners. Second, there are rules that assert that prisoners should avoid engaging in conflict, such as ``Do not fight with other inmates.'' Third, ``Do not exploit other inmates.'' This dictates that deceiving and fraud should be not tolerated against other upstanding convicts. Fourth, the inmate code asks felons not to weaken under any circumstances: ``Be strong.'' Fifth, there are many maxims that forbid convicts from cooperating with guards and authorities in the correction system in general, such as ``Do not trust the staff'' \citep[p. 525]{sutherland1992principles}. 

Those who lived by these rules were seen as ``good cons'' and generally enjoyed better reputations than prisoners who failed to comply with the code \citep{copes2013accounting,crewe2005codes,liebling2012prisonlife}. In a setting where physical threats are frequent, enacting the code gave convicts an advantage. There is also evidence that similar prescriptions are followed in other parts of the world, such as the United Kingdom, New Zealand, Mexico, Spain, and Thailand \citep{akers1977prisonization,sirisutthidacha2014patterns,winfree2002prisoner}. While not uniformly enforced in these countries, the inmate code apparently serves as a guide to felons abroad \citep[p. 843]{copes2013accounting}.

Scholars have proposed two theories to explain inmate culture and the origins of the convict code. The first is called the deprivation model. This theory suggests that inmate behavior is largely a product of prison life itself \citep{clemmer1940prison,irwin1980prisons,mccorkle1954resocialization}. According to this view, the convict code expresses a collective ``situational response'' \citep{akers1977prisonization} to the problems of ``prisonization'' \citep{clemmer1940prison}, that is, the deprivation of freedom, security, heterosexual relations, goods and services, and personal autonomy felons routinely endure \citep{sykes1958society}. The model also stresses that this feeling of deprivation is pervasive in jails, and to a varying extent all inmates are familiar with it. This shared experience is what binds prisoners together and it is the main reason why felons adopt the convict code.

Conversely, other authors contend that the convict code is merely an institutionalized version of the thieves' code. This theory is called the importation model and, as the name suggests, it states that criminals bring their former beliefs and behavior to jails   \citep{irwin1962thieves,irwin1970felon,irwin1980prisons}. A number of inmates come from neighborhoods with high levels of violence or notable presence of gangs; hence, it is not surprising that there are strong links between street subculture and the convict code \citep[p. 96]{sirisutthidacha2014patterns}.  \citet[p. 12]{irwin1980prisons} points out the many similarities: 

\begin{quote}
The central rule to the thieves' code was ``thou shalt not snitch.'' In prison, thieves converted that to the dual form of ``do not rat on another prisoner'' and ``do your own time.'' Thieves were also obliged by their code to be cool and though, that is to maintain respect and dignity; not to show weakness; to help other thieves; and to leave most prisoners alone.
\end{quote}

Nonetheless, these two hypothesis are not fundamentally incompatible, and scholars now agree that both factors help explain the emergence of the inmate culture \citep{schwartz1971pre,trammell2009values}. On the one hand, prisoners do not enter jails like a \textit{tabula rasa} as the deprivation model predicts. On the other hand, inmate behavior is also mediated by prison conditions. Whereas the exact causal mechanisms are yet to be specified \citep{delisi2004gang}, the convict code is likely a result of both social deprivation and previous criminal behavior. 

But regardless of its origins, over the past decades the inmate code has clearly declined in importance \citep{irwin1970felon,jacobs1975stratification,skarbek2014social}. This does not mean that the code's prescriptions are outdated: inmates continue to refer to them and often punish those who systematically violate the code's core tenets  \citep{copes2013accounting,trammell2012enforcing}. However, the growth of the American incarcerated population has significantly weakened the influence of old norms. The demographic shift has often been dramatic. California provides a relevant example. Between 1945 and 1970, the inmate population grew from 6,600 to about 25,000, and from 1950 to 2012 the number of prisons increased from 5 to 33 \citep{bass1975analysis,skarbek2014social}. This inflow of new prisoners indicates that spreading and enforcing the convict code became more costly than in the past. Consequently, young inmates are less likely to know and internalize such informal rules \citep{hunt1993change}. 

The expansion of the prison population has also diminished the influence of the inmate code through other channels. Norms are very effective at promoting coordination in small groups, but as the number of interactions increase, the opportunities for an individual to defect multiply \citep{bowles1998moral,cook2009whom}. In groups with loose social ties, people have only imperfect information about one other, so reputation effects are not a strong deterrent to uncooperative behavior \citep{shapiro1983premiums}. Furthermore, in a large community, individuals have additional incentives to free ride and let others bear the costs of punishing norm violators \citep{groves1977optimal,olson1965logic,samuelson1954pure}. Thus, a norm-based system such as the convict code tends to break down as the number of felons increase. 

The demand for protection in prisons has not declined with a growing inmate population. Rather, the opposite has occurred. However, inmates responded to this unprecedented situation by devising a new type of organization to provide order in jails. This is how prison gangs turned into powerful institutions. Prison gangs are well equipped to enforce rules in a large and heterogeneous penal system. As we noted in the previous section, gangs can monitor their members through the community responsibility system. Moreover, these groups provide valuable information to felons. Inmate organizations usually have rigid admission criteria, and they often publicize their acceptable standards of behavior in written documents  \citep{skarbek2010putting,skarbek2012prison}. This enhances cooperation as inmates know that prisoners who are affiliated with gangs are likely to be trustworthy. The affiliation process itself is already a costly and credible signal. Finally, prison gangs can mobilize a significant amount of money, violence and merchandise through their networks. The scale of their operations allows them to offer protection and material benefits to hundreds or even thousands of members \citep{blatchford2008black,camp1985prison,lessing2014build}. In an overcrowded penal system, these are all desirable qualities.

Prison gangs should be interpreted as an unintentional consequence of the massive demographic shift that has taken place in American prisons in the last years. This shift has made the previous system of norms, the convict code, insufficient to meet prisoners' demands for social order. Gangs provide security and facilitate trade in a diverse penal system by using effective enforcement mechanisms and transmitting reliable information to inmates. Prison gangs are therefore not a cause, but a solution to many of the inmates' problems.

\section{Prison Gangs and Street-Level Drug Markets}
\label{sec:outside} 

\noindent
The increase of the inmate population has also enabled prison gangs to exert considerable influence over criminals outside the penal system \citep{bony2015prison,dias2011pulverizaccao,lessing2014build,skarbek2011governance,valdez2005mexican}. This may seem puzzling at first. Inmates have limited access to goods and services, and they are under constant surveillance by prison officers. How do they manage to extort street drug dealers from behind bars?

Gangs often have long term goals. Inmate gangs are perpetual organizations, and many of them explicitly state that membership is for life \citep{blatchford2008black,mendoza2005mexican}. Such high entry barriers serve an important purpose for prison gangs. By demanding considerable upfront costs, gangs are able to select the most competent criminals and limit free riding \citep[p. 704]{freire2014entering,skarbek2011governance}. On the demand side, serious offenders also have good reasons to be affiliated with prison gangs. Since experienced criminals are more likely to serve lengthy sentences, they have a strong interest in securing long-term protection in jail.\footnote{There are a number of studies that associate prison gang membership with recidivism. See, among others,  \citet{fleisher2001going}, \citet{huebner2007gangs} and \citet{olson2004relationship}.} Survey evidence seems to confirm this hypothesis: \citet{fong1995blood} find that on average only about five percent of prison gang members leave these organizations. These two facts suggest that, in contrast with popular opinion, gangs are future-oriented clubs. 

In this regard, prison gangs are not qualitatively different from nation states. Like states, gangs are groups who have successfully monopolized coercive power over tracts of territory. Again like states, gangs need a constant inflow of resources to perpetuate themselves and provide services to their members. One way to do so is simply to plunder those under its domains. But while plundering may transfer wealth from the population to the ruling groups, the strategy only works for short periods. Individuals who live under ``extractive institutions'' tend to underproduce because they quickly realize their surplus will be seized by others \citep{acemoglu2012nations}. Instead, leaders are better off if they promote market-based institutions and extract taxes that are high enough to generate a profit to the ruler, but that are not so onerous that they discourage wealth generation. 

Borrowing Mancur Olson's terminology, one may say that prison gangs can earn more wealth in the long run by acting as a `stationary bandit' instead of a `roving' one. A stationary bandit is a leader who ``[\dots] has an encompassing interest in his domain that leads him to provide a peaceful order and other public goods that increase productivity'' \citep[p. 567]{olson1993dictatorship}. By creating social order in jails, gangs create an environment that is conducive to illegal trade; and the more money traders make, the richer a gang becomes by taxing them.

This reasoning can be extended to encompass street-level drug dealers as well. Nevertheless, jail-based extortion is only successful if street criminals anticipate incarceration. The causal mechanism is straightforward. Prisons are hostile settings, and often inmates are sent to facilities far away from their home turf. In such a scenario, lawbreakers cannot count on their extended network. Hence, if a criminal believes that he/she has a good chance of going to jail, it is reasonable to buy an `insurance policy' against potential harm caused by strangers. This is a service prison gangs can provide to street drug dealers. Since offenders currently have a very high likelihood of being incarcerated -- a result of America's harsh stance on crime -- it is important for them to secure some type of protection in advance. Thus, criminals outside the penal system have an incentive to pay `taxes' to prison gangs even \textit{before} they go to jail \citep{skarbek2011governance,skarbek2014social}. The likelihood of recidivism, in contrast, offers an incentive for criminals to continue paying gang fees after release. 

This is why the Mexican Mafia, also known as \textit{La Eme} (Spanish for ``the M''), has been able to extort drug dealers in Los Angeles. California prisons are notably overcrowded, so prisoners cannot rely exclusively on state officials for security \citep{muradyan2008california,newman2012brown}. Additionally, as it is one the largest gangs in California, the Mexican Mafia itself poses a credible threat of assault of individuals in jails. Incarceration and recidivism rates in California are also high: almost 50\% of all felons paroled in 1994 returned to jail within three years \citep{langan2002recidivism}. This indicates that criminals anticipate incarceration in that state. La Eme is also widely regarded as a powerful inmate organization, thus felons are aware that the gang has the required means to provide governance in jails and ameliorate prison conditions for its members \citep{rafael2013mexican,valdez2005mexican}. In a nutshell, only a minority of prisoners -- such as those who did not expect incarceration, who are already affiliated with other groups or live in a jurisdiction over which La Eme has no control -- would not hire protection from the Mexican Mafia \citep[p. 709--710]{skarbek2011governance}. Paying gang taxes is a rational choice for many street criminals in California.

The mechanism described above explains why California prison gangs have progressively projected power onto the streets. A worrying outcome of this process is that the growing influence of prison gangs undermines state authority in the long run. Since inmate groups now control a larger criminal population and can extract more taxes from illegal activities, gangs are in a better position to demand concessions from public authorities and even orchestrate attacks against state targets \citep{lessing2014build}. Mass incarceration, in this sense, has strengthened prison gangs' position behind and beyond bars.

\section{Inmate Groups Worldwide}
\label{sec:global} 

\noindent
As previously mentioned, inmate organizations are not only found in the United States. Prison gangs control the penal system in many countries, both in developing and developed nations. However, the extent to which they govern jails worldwide varies considerably. In regions like Latin America, where state authority is inefficient and the number of convicts high, most prison governance is supplied by private actors \citep{biondi2010junto,darke2013inmate,dias2009guerra,lessing2014build,skarbek2010self}. In contrast, informal institutions are relatively unimportant in Scandinavian countries \citep{larsson2011svensk,pratt2008scandinavian}. Although gangs do exist in Scandinavia, they are not particularly influential because state presence is robust. Social order in prisons is provided by official channels. Apart from the quality of state governance, the size and demographics of the inmate population also determine the scope of informal institutions. Large, diverse inmate populations are likely to be organized through gangs, whereas small and/or homogeneous groups are usually governed by decentralized mechanisms such as norms and rules. We exploit the patterns of inmate population in California and England to explain these differences. We start with a brief analysis of how state provision impacts the formation of prison groups, then we discuss the role of inmate demographics on inmate organizations.

Prisons in Latin America are severely understaffed and underfunded. In Brazil, there is one member of staff for 5.6 prisoners on average; but since a most of them are employed in administrative tasks, these numbers overestimate the true overseeing capabilities of the state \citep[p. 100]{mariner1998behind}. In the northeastern state of Rio Grande do Norte, for instance, three guards were reported overseeing 646 inmates in a prison \citep[p. 102]{mariner1998behind}. Lack of personnel is even more acute in Rio de Janeiro. \citet[p. 275]{darke2013inmate} reports that in one local jail there were only six officials in charge of 1,405 convicts. Inmates often have no access to basic items such as toiletries, mattresses, and hot water \citep[p. 65]{cnmp2013}. All these facts point out that official governance is in short supply in the country.

In Bolivia, prisoners face similarly harsh conditions. San Pedro Prison, a large penitentiary in the capital La Paz, was originally built to house 250 members, but now holds between 1,300 and 1,500 \citep{bbc2009carcel,cerbini2012sanpedro}. Poor provision of basic services is also common. The International Committee of the Red Cross affirms that many Bolivian prisoners get contagious diseases while in state custody, and one of their doctors claimed that Bolivian jails are ``an ideal breeding ground for tuberculosis'' \citep{vuilleumier2010latin}. State supervision is virtually non-existent, and one journalist once described how the guards are ``cowed, outnumbered, or corrupt enough that their goal is merely to keep the inmates in, and leave maintaining the prison to [the prisoners]'' \citep{gassaway2004san}. 

Since state action is limited, inmates are called upon to mediate all sorts of disputes within prison. Extralegal institutions have to provide an array of services in Latin America -- from protection to trade, from health services to social assistance -- and this is an important reason why gangs tend to be highly organized in the region. Brazil hosts powerful prison gangs, such as Rio de Janeiros's \textit{Comando Vermelho} (`Red Command'), formed in 1979 \citep{amorim2003cv,da2001quatrocentos}, and São Paulo's \textit{Primeiro Comando da Capital} (`First Command of the Capital'), created in 1993 \citep{biondi2010junto,dias2009guerra,souza2007pcc}. Both groups are notably powerful inside and outside prison walls, and largely control drug trade and enforce rules in their territories. In the case of Bolivia's San Pedro Prison, inmates regularly elect representatives that have a mandate to establish order and guarantee property rights in the penitentiary \citep{bbc2009carcel,skarbek2010self}. 

This stands in stark contrast to what we observe in Scandinavia. In Denmark and Sweden prison staff members outnumber inmates, and occupancy rates are always below the limit of 100\%.\footnote{See: \href{http://www.prisonstudies.org/world-prison-brief}{http://www.prisonstudies.org/world-prison-brief}. Access: 04/01/2016.} Prisons in Scandinavia are also small. The average number of inmates in correctional facilities in Norway, Denmark, and Sweden is 88, 70, and 68 respectively \citep{pratt2008scandinavian}. With a reduced inmate population and an efficient provision of official governance, the demand for private protection was never a great concern for prisoners in those countries. Prison gangs are rare in Scandinavia, but Sweden's \textit{Brödraskapet} (`The Brotherhood') is probably the most significant example of a local inmate organization \citep{larsson2011svensk}. Yet, the gang has never caused extensive damage to the prison system itself nor does it pose a significant menace to society at large, as Latin American inmate gangs currently do \citep{dias2009guerra,lessing2014build,wolf2010maras,wolf2012mara}.

Additionally, the influence prison gangs have in a given penal system is also conditioned by the characteristics of the local inmate population. In California, racial segregation has historically played a significant role in gang formation and recruitment \citep{blatchford2008black,camp1985prison,davidson1974chicano,skarbek2011governance}. As we have argued above, the rise in incarceration rates in California has diminished the importance of the system of norms that used to regulate inmate behavior. Coordination costs are higher for larger groups, therefore other types of organization -- prison gangs -- have become the major source of social order in Californian prisons. 

Evidence from England confirms this argument. There has been a growing influence of Muslim groups in the United Kingdom over the last decades \citep{hm2010muslim}, but they are far from being as powerful as gangs in California. In many facilities in England and Wales, gangs do not even exist, thus the convict code remains the major source of norms for inmates \citep{crewe2009prisoner,liebling2004prisons}. Finally, prison gangs in the United Kingdom lack permanent and mutually-exclusive group membership like they have in California. In this regard, although both United States and the United Kingdom are developed nations with similar quality of official governance, inmate demographics have generated different outcomes when it comes to extralegal organizations. Whereas in California prison gangs are providers and enforcers of rules, in England and Wales decentralized governance is still common.

As we have seen, inmate organizations govern large parts of the world, but they do not do it uniformly. Prison governance methods vary both over time and across space. Nations with a vast inmate population tend to have stronger prison gangs than those in which the number of convicts is small. Ethnic heterogeneity is also conducive to gang formation, as is low-quality state provision of goods and services.  

\section{Conclusion}
\label{sec:conclusion} 

In this chapter, we have offered a brief overview of the current literature on prison gangs. We have discussed how inmate gangs are a rational response to several challenges of prison life, and how they promote illegal trade and provide security to inmates. We have also argued that prison gangs sustain internal order through a community responsibility system, and discussed how this system fosters trust between inmate communities. Moreover, in Section Three we presented the main rules of the convict code and explained why the code has decreased in importance over the past few years. These are the conditions which allowed gangs to increase their dominance behind bars and later to expand their protection services to street criminals. Finally, we have shown that the quality of state governance and inmate demographics are the main factors that account for the different levels of prison gang governance around the world.

However, there are many under-researched topics in the prison gang literature. Although there are several relevant academic works on gang formation, comparative studies are still uncommon in the field. Testing causal mechanisms in a range of prison gangs could help scholars isolate the necessary and sufficient conditions for gang formation and development. This would require collecting and standardizing large sample data, an effort that is yet to be done but which would be fruitful for scholarship on prison gangs \citep{fleisher2001overview}.

Illicit markets in prisons are also not well understood. Thus far, there are only a small number of studies about how prisoners engage in trade, mostly focused on the inmates' demand for drugs and other goods behind bars. Little is known about how drug dealers establish their networks, and how hard drugs determine other aspects of prison life such as internal hierarchies or inmate financing and credit tools \citep[p. 348]{crewe2006prison}. 

Another topic which deserves further attention is how inmate groups decide their ``repertoire of violence,'' that is, which type of violence prison gangs use against their own members or non-affiliated convicts. Whereas some gangs employ physical threats only as a last resort \citep{crewe2006prison,trammell2012enforcing}, others make extensive use of violence as a means to enforce rules \citep{dias2009guerra,lessing2014build}. In prisons, violence also has a clearly communicative purpose \citep{gambetta2009codes}, so comparing and analyzing violence strategies would enable us to gain a better grasp of gangs' relative positions in the inmate community and to analyze how these groups manage (or fail) to influence others' decisions.

Finally, the relationship between the state and prison gangs can be further explored by scholars. It is important to know under what conditions the state chooses to confront, appease or collude with an inmate group. It is important to identify which mechanisms lead the state to adopt different approaches when dealing with extralegal groups. As mass incarceration has become one of the most pressing issues not only in the United States but also abroad, a call for evidence-based policies seems timely. Taken together, these efforts will allow researchers and policy makers to formulate better, more efficient approaches for managing prison gangs.


\newpage

\bibliographystyle{apalike}
\bibliography{references}


\end{document}
